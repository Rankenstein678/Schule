\documentclass{article}
\usepackage[T1]{fontenc}
\usepackage{lmodern}

\begin{document}
\section*{Muskeln}
\subsection*{Aufbau}
Der Muskel ist von einer Schicht an Bindegewebe umgeben. In diesem Bindegewebe (Faszie) befinden 
sich ebenfalls von Faszien umgebene Muskelfaserbündel. Diese Faserbündel bestehen aus einzelnen Muskelfasern,
welche wiederum aus Fibrillen bestehen. Die Fibrillen sind von Sarkoplasma umgeben und bestehen aus aneinadergereihten 
Sarkomeren. Diese sind die krontraktilen Elemente der Muskeln und bestehen aus dünnen Aktin und dicken Myosinfilamenten.

\subsubsection*{Faserarten}
I slow twitch -> langsam kontraktierend + ermüdend, dunkel, für aeroben Stoffwechsel optimiert

II fast twitch ->  schnell kontraktierend + ermüdend, hell, für aneroben Stoffwechsel optimiert

IIa -> höhere Intensität als I, keine rasche Ermüdung

IIx -> sehr hohe Intensität, schnelle Ermüdung

\subsection*{Motorische Einheiten}
Eine Motorische Einheit ist ein Motorneuron und ihre zugehörige Muskelfaser.

intramuskuläre Koordination: Mehrere motorische Einheiten werden gleichzeitg aktiviert

intermuskuläre Koordination: Zusammenspiel mehrerer Muskeln im Rahmen einer Bewegung

Agonist: kontrahierender Muskel - Antagonist: sich dehnender Muskel

Es gibt immer Agonist und zugehörigen Antagonist

\subsection*{Auf welche Arten arbeiten Muskeln}
\subsubsection*{statisch}
Muskel kontrahiert sich bei gleichbleibender Muskellänge und Zunahme der Spannung (verharrend)
\subsubsection*{konzentrisch}
Muskel kontrahiert bei verkürzender Muskellänge und Erhöhung der Spannung (überwindend)
\subsubsection*{exzentrisch}
Muskel dehnt sich bei verlängerter Muskellänge und Erhöhung der Spannung (nachgebend)

\section*{Kraft}
Kraft im biologischen Sinne ist die Fähigkeit des Nerv-Muskel-Systems, durch Muskelkontraktion Widerstände
zu überwinden, halten oder ihnen entgegenzuwirken.
\subsection*{Kraftarten}
Man unterscheidet verschiedene Kraftarten. Diese treten immer in Mischformen auf.
\begin{itemize}
    \item Maximalkraft: größtmöglichste Kraft, die gegen einen Widerstand gerichtet werden kann
    \item Kraftausdauer: Ermüdungswiderstandsfähigkeit bei anhaltender oder sich wiederholender Belastung
    \item Schnellkraft: optimal schnell Kraft bilden und die Maximalkraft erreichen
    \item Reaktivkraft: Kraftstoß erzeugen
\end{itemize}

Die Maximalkraft ist die Vorraussetzung für das Entwickeln der anderen Kräfte. Sie hängt von 3 Komponenten ab:
\begin{enumerate}
    \item Muskelquerschnitt
    \item intermuskuläre Koordination
    \item intramuskuläre Koordination
\end{enumerate}
\section*{Training}
\subsection*{Maximalkraft}
\begin{itemize}
    \item Muskelaufbautraining: mittlere bis hohe Intensität und hohe Wiederholungsratem
    \item Intramuskuläres Koordinationstraining: hohe-höchste Intensotät, geringe Wiederholungszahlen
\end{itemize}
\subsection*{Schnellkraft}
mittlere Belastung + Wiederholung, maximale Geschwindigkeit
\subsection*{Reaktivkraft}
sehr hohe Intensität (100\%)
\subsection*{Kraftausdauer}
geringe bis mittlere Intensität, Wiederholung bis zur Erschöpfung
\section*{Trainingshinweise}
Maximalkraft steigern, Kräftegleichgewicht zwischen Agonist und Antagonist, langsame Bewegung
\end{document}