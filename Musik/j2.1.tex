\documentclass{article}

\usepackage[ngerman]{babel}

\begin{document}
\section*{Tonleitern oder Whatever}
\subsection*{Die chromatische Tonleiter}
Bei der chromatischen Tonleiter sind die Töne immer einen Halbtonschritt
voneinander entfernt.

Bass-Schlüssel: Zwischen den Punkten ist f
\section*{Mittelalter}
\subsection*{Gregorianik}
Noch vor der Mehrstimmigkeit gab es den Gregorianischen
Choral, der einstimmig und unbegleitet auf Latein gesungen wurde. Er war die
offizielle Musik des Gottesdiensts.

\subsection*{Notenschrift}
Ein erster Versuch Musik aufzuschreiben waren die Neumen.
Sie stellten die Dirigierbewegungen dar.
Die Notenschrift von Guido von Arezo macht es möglich
mehrstimmige Musik genau aufzuschreiben

\subsection*{Mehrstimmigkeit}
In der geistlichen Musik gab es 3 Arten der Mehrstimmigkeit
\subsubsection*{Das schwebende Organum}
Beim schwebenden Organon singt eine Stimme einen Grundton
während eine andere die Melodie singt.
\subsubsection*{Das Parallel Organum}
Beim Parallel Organon werden beide Stimmen im Melodieverlauf
parallel geführt. Dabei sind sie eine Quarte auseinander.
\subsubsection*{Diskantus}
Beim Diskantus laufen die Stimmen aus - und gegeneinander.
\section*{Die Zwölftonreihe}
Bei der Zwölftonreihe werden die Töne einer chromatischen Tonleiter
in einer beliebigen Reihe angeordnet. Alle Töne sind gleichwertig,
das heißt, dass alle Töne gespielt werden müssen, bis sich ein Ton
wiederholen darf. Um Variation möglich zu machen sind gewisse
Transformationen erlaubt.
\subsubsection*{Der Krebs}
Beim Krebs wird die Reihe rückwärts geschrieben
\subsubsection*{Die Transposition}
Bei der Transposition wird die Reihe um ein oder mehrere
Halbtonschritte nach oben oder unten verschoben.
\subsubsection*{Die Umkehrung}
Die Intervalle werden umgekehrt, d.h. 3 HS nach oben wird zu
3 HS nach unten.
\subsubsection*{Die Krebsumkehrung}
Krebs + Umkehrung

KU == UK
\subsubsection*{Die Komplementär Methode}
Alle Intervalle werden in ihr komplementär Intervall umgewandelt.
NHS = 12 - HS

\section*{Renaissance}
Das Klangideal der Renaissance war eine klare leicht
verständliche Musik. Komponisten wie \emph{Giovanni Palestrina}
kombierten dieses Ideal mit der mehrstimmigen, polyphonen Musik des 
Mittelalters. Seine Musik hinderte die Kardinäle auf dem Konzil von Trient
daran mehrstimmige Kirchenmusik zu verbieten.
\subsection*{Die verschiedenen Formen des Volkslieds}
\subsubsection*{Motette}
Im Gegensatz zum Choral, in dem der Sopran führt, sind hier alle Stimmen
gleichberechtigt. (Choral: homophon <> Motette: polyphon) Eine Motette ist
ernster als ein Quodlibet.
\subsubsection*{Quodlibet}
Beim Quodlibet (Spielerei, durcheinander, wörtl. "Was beliebt") singt jede
Stimme ihren eigenen Text.
\subsubsection*{Ricercar}
Singstimmen einer Motette werden durch Instrumente ersetzt. Dafür wurden 
Instrumentenfamilien gebaut. (Alt-Horn, Sopran-Horn, usw.)
\subsubsection*{Canzone}
Verschiendene Instrumentenfamilien spielen zusammen (+ Gesang als Nebensache)

\section*{Moderne}
Lange Zeit waren Töne nach dem System der Tonalität geordnet, d.h. alle Töne 
beziehen sich auf einen Grundton 
\end{document}