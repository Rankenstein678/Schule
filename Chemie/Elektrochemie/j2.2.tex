\documentclass{article}

\usepackage{graphicx}
\usepackage{gensymb}


\makeatletter
\newcommand{\rmnum}[1]{\romannumeral #1}
\newcommand{\Rmnum}[1]{\expandafter\@slowromancap\romannumeral #1@}
\makeatother


\begin{document}
\section*{Oxidationszahlen}
Oxidationszahlen werden mit der Notation +/- Römische Zahl geschrieben.
Bei Organischen Molekülen sollten die Moleküle auf jeden Fall gezeichnet werden.
Bei anorganishcen kann es auch reichen nach den Faustregeln vorzugehen:
\begin{enumerate}
    \item Wasserstoff +\Rmnum{1}
    \item Fluor -\Rmnum{1}
    \item Sauerstoff -\Rmnum{2}
    \item Halogene -\Rmnum{1}
\end{enumerate}

\section*{RedOx Reaktionen}
Bei RedOx Reaktionen werden Elektronen nach einem Donator-Akzeptor-Prinzip übertragen.
Der Donator wird oxidiert, während der Akzeptor reduziert wird. Bei einer RedOx Reaktion
muss immer beides gleichzeitig stattfinden. Eine RedOx Reaktion lässt sich an den Oxidationszahlen
erkennen. Die OZ des reduzierte Stoffs wird reduziert, während die des oxidierten steigt.

Die Schreibeweise für ein korrespondierendes RedOx Paar ist: reduzierte Form/oxidierte Form, z.B. $Cu/Cu^{2+}$
oder $2 Cl^-/Cl_2$

\subsection*{Die RedOx Reihe}
RedOx Reaktion laufen nur in eine Richtung spontan ab. Daher lassen sie sich in eine Reihe einordnen.

\includegraphics*[width=\textwidth]{img/Redoxreihe}

\section*{Galvanische Zellen}
\includegraphics*[width=\textwidth]{img/galvanische_zelle.jpg}

\subsection*{Lösungstension}
In einer Lösung gehen Stoffe häufig als Ionen in Lösung. Dabei hinterlassen sie Elektronen. Das ganze ist
eine Gleichgewichtsreaktion
    \begin{equation}
        Cu(s) \rightleftharpoons Cu^{2+}(aq) + 2 e^-
    \end{equation}
\subsection*{Konzentrationselement}
Bei einem Konzentrationselement werden 2 gleiche Halbzellen verwendet. Der einzige Unterschied 
ist die Konzentration der Salzlösung. Ist die Konzentration geringer, so verschiebt sich die Konzentration
nach ''rechts''. DAher wird lädt sich die Elektrode stärker auf und das Potential wird kleiner.
Pro Größenordnung unter 1 mol/L sinkt das Potenzial um 0,059 Volt.

\subsection*{Die Standardwasserstoffhalbzelle}
Dem RedOx Paar $H_2/2H^+$ der Standardwasserstoffhalbzelle wird als Referenz das Elektrodenpotenzial 0V zugeordnet
In der Halbzelle befindet sich eine Salzsäure Lösung mit der Konzentration c = 1 mol/L. Es wird Wasserstoff
bei einer Temperatur von 25\textdegree C mit 1013 hPa unter eine platinierte Platinelektrode gesprudelt.

Anhand der Wasserstoffhalbzelle kann jeder Halbzelle ein Standardelektrodenpotenzial zugeordnet werden:
\begin{equation}
    U = \Delta E\degree = \Delta E\degree (Akzeptor Halbzelle) - \Delta E\degree (Donator Halbzelle)    
\end{equation}

\section*{Elektrolyse}
Bei einer Elektrolyse wird das Prinzip einer Galvanischen Zelle umgekehrt. 
Sie ist die Umkehrreaktion der spontan ablaufenden Reaktion in der GZ.
\subsection*{Zersetzungsspannung}
Für eine kontinuierliche Abscheidung von Elektrolyseprodukten ist eine Mindestspannung, die Zersetzungsspannung
$U_Z$ nötig. Bereits bei geringen Spannungen entstehen Elektrolyseprodukte. Somit ist eine GZ und eine Gegenspannung 
entsteht. Trotzdem fließt unterhalb von $U_Z$ bereits ein geringer Strom, auch \emph{Diffusionsstrom} genannt.
Dies geschieht, da kleine Teile der Elektrolyseprodukte in Lösung gehen und damit für die Rückreaktion ausfallen..

Die Differenz zwischen tatsächlicher und theoretischer Spannung heißt Überspannung $U_{\ddot{u}}$

\begin{equation}
    U_{\ddot{u}} = U_{gemessen} - U_{theoretisch}
\end{equation}

Es läuft stehts die Reaktion mit der kleinsten Zersetzungsspannung ab


\end{document}


