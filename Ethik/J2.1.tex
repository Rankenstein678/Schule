\documentclass{article}
\usepackage[ngerman]{babel}


\begin{document}
\section*{Der gute Wille}
Intellektuelle Fähigkeiten, wie Intelligenz und Verstand, sowie natürliche Veranlagungen, wie Mut und Entschlossenheit,
sind zwar wünschenswert, jedoch können sie zu unlauteren Zwecken missbraucht werden, sollte der Mensch, der sie nutzt,
einen bösen Charakter hat. Auch Dinge, die wir äußeren Umständen verdanken, wie Macht, Reichtum oder Gesundheit können 
einerseits selbstbewusst machen, jedoch können sie, wenn kein guter Wille vorhanden ist, der ihren Einfluss zügelt, zu
Überheblichkeit führen und die Handlungen eines Menschen ins negative beeinflussen. Und selbst Eigenschaften wie Besonnenheit,
die sich nicht ins negative drehen lassen können ohne einen guten Willen missbraucht werden. (beispielsweise durch einen
Kriminellen)

Der gute Wille ist somit \emph{an sich gut}. Was durch von diesem Willen ausgelöste Handlungen passiert ist für die moralische 
Bewertung völlig irrelevant.

\section*{Aufklärung}
Nach Kant ist die Aufklärung ''der Ausgang des Menschen aus seiner selbstverschuldeten Unmündigkeit''. Unmündigkeit bedeutet
ohne die Leitung eines anderen nicht frei denken zu können. Selbstverschuldet ist diese Unmündigkeit nur, wenn ihr Ursprung
nicht in einem Mangel des Verstandes, sondern in einem des Muts liegt. Faulheit und Feigheit sind die Ursachen, warum so viele
Menschen trotz der eigenen Fähigkeit frei zu denken, zeitlebens unmündig bleiben. Wer bezahlen kann braucht nicht zu denken.
So übernimmt für viele Menschen ein Buch den Verstand und ein Seelsorger das Gewissen.

\subsection*{Die oberste Pflicht}
Es ist die oberste Pflicht jedes Menschen sich selbst auf moralische Volkommenheit zu erforschen. Dies heißt herauszufinden,
ob das eigene Herz gut oder böse sei, die Quelle der Handlungen lauter oder unlauter und was ursprünglich Teil einer selbst 
war und was erlernt wurde.
Die moralische Selbsterkenntnis (oder auch Höllenfahrt der Selbsterkenntnis) ist der erste Schritt aller menschlichen Weisheit.
Um diese zu erreichen müssen Hindernisse des inneren bösen Willens weggeräumt und ein guter Wile entwickelt werden.

\section*{Der kategorische Imperativ}
Der Wille ist das Vermögen, nach Prinzipien zu handeln und das zu wählen, was die Vernunft unabhängig von individuellen 
Neigungen als praktisch notwendig, d.h gut erkennt. Da der Willen in der Realität jedoch Trieben unterworfen ist und
nicht der Vernunft folgt, muss das Prinzip einer Vernunft, die für jeden gilt als ein \emph{Imperativ} formuliert werden.

Praktisch gut ist was durch eine Vorstellung der Vernunft, d.h nicht aus Ursachen, die für jeden verschieden sein können,
sondern aus Gründen, die für jedes vernünftige Wesen gelten, den Willen bestimmt.

Ist eine Handlung als Mittel zur Erreichung eines anderen Zwecks gut, so wird sie dem Willen durch einen hypothetischen
(bedingten) Imperativ vorgestellt. Wird die Handlung als an sich selbst gut vorgestellt, so geschieht dies durch einen
\emph{kategorischen} (unbedingt, ohne Wenn und Aber) Imperativ.

\subsection*{Prinzip der Vernunft}
Die Folgen einer Handlung aus Klugheit sind niemals vorhersebar und immer auf die Folgen für einen selbst bedacht.
Daraus folgt, dass es Klüger ist nach einer Maxime zu handeln.
$\rightarrow$
Die Handlung aus Pflicht enthält ein Gesetz für mich. Daher muss ich nicht alle Folgen bedenken $\rightarrow$
Die Handlung selbst ist gut.

\subsection*{Maximentestverfahren}
\begin{enumerate}
    \item Feststellung der konkreten Handlung (Deskriptiv):
    Bsp: ''Ein anderer sieht sich in der Not gedrungen, Geld zu borgen, von dem er weiß, dass er es nie zurückzahlen kann.''
    \item Formulierung der Handlungsmaxime:
    Der subjektive Handlungsgrundsatz (Maxime), wekcher der beabsichtigten Handlung zugrunde liegt, in der Form
    ''Immer wenn ..., dann werde ich ...'':
    Bsp: ''Immer wenn ich mich in Geldnot befinde, so will ich Geld vorgen und versprechen es zurückzuzahlen, obwohl
    ich weiß, dass ich dies nicht können werde.''
    \item Verallgemeinerung (Universalisierung) der Handlungsmaxime:
    (Hypothetische) Verallgemeinerung der subjektiven Handlungsmaxime zu einem objektiv (allgemein) gültigen
    Handlungsgesetz:
    Bsp: ''Immer wenn sich jemand in Geldnot bedindet, borgt er sich Geld, obwohl er wuß, dass er es nicht zurückzahlen kann.''
    \item Moralische Beurteilung der Maxime: Kriterium Widerspruchsfreiheit:
    Kann ich die Verallgemeinerung widerspruchsfrei denken und wollen
\end{enumerate}
\end{document}