\documentclass{article}

\usepackage[ngerman]{babel}
\usepackage[a4paper,margin=1in,footskip=0.25in]{geometry}

\begin{document}
\section*{Theorien der internationalen Politik}
\subsection*{Realismus}
Im Realismus wir die Politik im anarchischen Raum durch die existenzielle Unsicherheit der Staaten bestimmt.
Internationale Politik ist ein für die Staaten überlebenswichtiger Machtkampf.
\begin{center}
    \begin{tabular}{p{0.5\textwidth} p{0.5\textwidth}}
        Akteure                               & Staaten und Militär                                                        \\
        Disposition                           & Egoismus \& Zweckrationalismus                                             \\
        Einschränkung der Anarchie möglich    & Nein, Anarchie wir immer wieder reproduziert                               \\
        Bedingungen für Friede \& Kooperation & Kooperation nur temporär                                                   \\
        Antrieb internat. Pol.                & existenzielle Unsicherheit $\rightarrow$ Machtstreben, Sicherheitsbedenken
    \end{tabular}
\end{center}
\section*{Institutionalismus}
Im Institutionalismus ist ein treibender Faktor die Interdependenz der Staaten.
Durch wechselseitige Abhängigkeiten verringert sich der Nutzen von Gewalt und der Bedarf nach Kooperation
wächst. Dadurch enstehen internationale Organisationen, die Regime (Regelwerke) festlegen.
\begin{center}
    \begin{tabular}{p{0.5\textwidth} p{0.5\textwidth}}
        Akteure                               & Staaten \& intenationale Organisationen                                    \\
        Disposition                           & Gewinnstreben, Egoismus, Zweckrationalismus                                \\
        Einschränkung der Anarchie möglich    & Anarchie kann eingeschränkt / zivilisiert werden, nicht jedoch abgeschafft \\
        Bedingungen für Friede \& Kooperation & Interdependenz, Regeln, Überwachung und Strafen                            \\
        Antrieb internat. Pol.                & Streben nach abs. Gewinnen $\rightarrow$ Kooperationswillen
    \end{tabular}
\end{center}
\section*{Konstruktivismus}
Es bilden sich internationale Gemeinschaften, deren Mitglieder eine Kultur teilen. Durch gemeinsame Werte und Normen
entwickeln sich freundschaftliche Verhältnisse und Konflikte können argumentativ gelöst werden. Verschiedene
Wertegemeinschaften können jedoch nur schwer miteinander interagieren. Hier verstärkt sich die Anarchie sogar noch.
\begin{center}
\begin{tabular}{p{0.5\textwidth} p{0.5\textwidth}}
    Akteure                               & Staaten, Wertegemeinschaften                                    \\
    Disposition                           & Normen / Werte                                \\
    Einschränkung der Anarchie möglich    & innerhalb der Gemeinschaften kann die Anarchie überwunden werden. Außerhalb tritt sie verschärft auf. \\
    Bedingungen für Friede \& Kooperation & gemeinsame Werte, gesellschaftliche und kulturelle Ansichten                            \\
    Antrieb internat. Pol.                & Wunsch Kultur zu vertiefen \& gemeinsame Werte
\end{tabular}
\end{center}
\section*{Der erweiterte Sicherheitsbegriff}
Seit der Gründung hat sich der Begriff ''Sicherheit'' stark verändert. Während früher nur die Abwehr militärische Angriffe anderer Staaten
als Teil der nationalen Sicherheit erfasst wurden, versteht man heute unter dem Begriff viel mehr. Der Sicherheitsbegriff wurde in 
vier Dimensionen erweitert:
\subsection*{Sachdimension}
Die Sachdimension bezeichnet den Problembereich, in den das politische Problem fällt. Die Aspekte sind
\begin{enumerate}
    \item militärisch
    \item ökonomisch
    \item ökologisch
    \item humanitär
\end{enumerate}
\subsection*{Referenzdimension}
Die Referenzdimension befasst sich mit der Frage, wessen Sicherheit gewährleistet werden soll.
\begin{enumerate}
    \item Staat
    \item Gesellschaft
    \item Individuum
\end{enumerate} 
\subsection*{Raumdimension}
Die Raumdimension betrifft die Frage für welches Gebiet Sicherheit angestrebt wird.
\begin{enumerate}
    \item national
    \item regional
    \item international
    \item global
\end{enumerate}
\subsection*{Gefahrdimension}
Die Gefahrdimension beschreibt wie das Problem verstanden wird.
\begin{enumerate}
    \item Bedrohung
    \item Verwundbarkeit
    \item Risiko
\end{enumerate}
\section*{Die UN - Handlungsmöglichkeiten und die Schutzverantwortung}
\subsection*{Ziele der UN}
\begin{itemize}
    \item Weltfrieden und internat. Sicherheit wahren
    \item freundschaftliche Beziehungen auf Grundlage von Gleichberechtigung und Selbstbestimmung fördern
    \item Menschenrechte achten
\end{itemize}
$\rightarrow$ Frieden
\subsection*{Grundsätze}
\begin{itemize}
    \item Gleicheit der Mitglieder, Achtung der Menschenrechte
    \item Verpflichtunf zur friedlichen Streitbeilegung
    \item Verbot der Androhung \& Anwendung von Gewalt
    \item Beistandspflicht bei Maßnahmen der UN
    \item Interventionsverbot bei inneren Angelegenheiten
\end{itemize}
\subsection*{Hauptorgane}
Die UN ist in drei Hauptorganen organisiert:
\subsection*{Generalversammlung}
Die Generalversammlung ist die Versammlung aller Mitglieder der UN. Jeder Staat besitzt eine Stimme.
Meist reicht eine einfache Mehrheit, um ein Entscheidung zu fällen. Ausnahmen sind die Wahl der
nicht ständigen Mitglieder des Sicherheitsrats und die Aufnahme neuer Mitglieder, für welche eine 
2/3 Mehrheit benötigt wird. Intern funktioniert die Generalversammlung als Legislative, extern haben
ihre Resolutionen nur den Charakter von Empfehlungen.
\subsubsection*{Der Sicherheitsrat}
Der Sicherheitsrat besteht aus 15 Mitgliedern - 5 ständigen (den P5 China, Russalnd, Frankreich, UK \& USA)
und 10 nicht ständigen, welche auf 2 Jahre von der Generalversammlung gewählt werden. Jedes Jahr werden 5
neue bestimmt. Für eine Entscheidung ist ein 9/15 Mehrheit nötig. Die ständigen Mitglieder besitzen ein Vetorecht.
Die Aufgabe des Sicherheitsrats ist die Wahrung des Weltfriedens. Dazu ist er auch befähigt Gewalt einzusetzen.
\subsubsection*{Verwaltung}
Das Sekretariat regelt die Verwaltung. Ein Generalsekretär wird auf Vorschlag des SIcherheitsrats von der Generalversammlung
gewählt. Er regiert 5 Jahre und kann 1 mal wiedergewählt werden. Er koordiniert die Arbeit zwischen den Hauptorganen,
stellt Haushltspläne auf, repräsentiert die UN und lenkt die Aufmerksamkeit des Sicherheitsrats auf relevante Themen
\subsection*{Handlungsmöglichkeiten}
\subsubsection*{Friedliche Beilegung}
Vermitlungsvorschlag es Sicherheitsrats auf Ersuchen der Konfliktparteien.
\subsubsection*{Zwangsmaßnahmen}
\begin{itemize}
\item nicht militärische Sanktionen (Wirtschaftsboykott, Unterbrechen der dipl. Beziehungen, Unterbr. von Vehrkerswegen)
\item militärische Handlungen zur Erzwingung des Friedens (Demonstration, Blokaden, etc.)
\subsubsection*{Durchschlagskraft des Sicherheitsrats}
\begin{itemize}
    \item Weisung und Entschidungsbefugnis
    \item Feststellung der eigenen Zuständigkeit
\end{itemize} 
\end{itemize}
\subsubsection*{Prinzipien}
Primat der Souveränität \& zwischenstaatliche Lösungen vs. Primat der kollektiven Sicherheit 
\end{document}