\documentclass{article}
\usepackage[ngerman]{babel}
\begin{document}
\section*{Plato}
\subsection*{Ideenlehre}
Nach Plato ist allein die Idee wahr. Alles andere ist nur eine Erscheinung. Das Höhlengleichnis verdeutlicht
den Unterschied zwischen Wirklichkeit und Erscheinung.
\subsection*{Höhlengleichnis}
\subsubsection*{Aufbau}
Menschen, welche von Kindheit an, an Hals und Schenkeln gefesselt sind, sind in eine unterirdischen Höhle gefangen.
Durch ihre Fesseln sind sie gezwungen, auf eine Wand zu starren. Licht erhalten sie durch Feuer von oben und hinten.
Zwischen dem hinteren Feuer und ihnen, verläuft ein Weg mit Menschen, die allerlei Dinge tun und sich generell normal
verhalten. Die Gefangenen sehen nur die Schatten der Personen und ihre eigenen. In ihrer Realität, sind die Schatten
die Wirklichkeit. Das Weltbild der Gefangenen ist konstruktivistisch.

Wird nun einer der Gefangenen entfesselt und gezwungen ans Licht zu gehen, so wird er dort zunächst nichts erkennen
können. Mit der Zeit, wird er zuerst Schatten, dann den Nachthimmel und als letztes sogar die Sonne erkennen können.
Reflexionen zu erkennen sollte ihm leichter fallen, als die echten Dinge. Mit seiner neuen Wahrnehmung der Dinge,
wird er sich glücklicher schätzen, als seine Mitgefangenen. Steigt er jedoch wieder in die Höhle hinab, so wird er
die Schatten schlechter ausmachen können, als zuvor. Die anderen werden seinen Erzählungen nicht glauben und sich
in ihrer Wirklichkeit überlegen fühlen. Sie könnten sogar schlussfolgern, dass jeder, der nach oben geht getötet gehört.

\section*{Rene Descartes (Rationalismus)}
\subsection*{Meditation I}
\begin{itemize}
    \item alles, was er bisher dachte, in Zeifel ziehen und sein Denken von Grund auf neu aufbauen
    \item $\rightarrow$ Zweifel an der Sinneswahrnehmung
    \item alles könnte ein Traum sein
    \item ein Dämon/Täuschergott könnte ihn täuschen
    \item $\rightarrow$ mathematische Eigenschaften sind unzweifelhaft
\end{itemize}
\subsection*{Mediation II}
\begin{itemize}
    \item aus seinem Nachdenken schließt Descartes ''ich bin, ich existiere'' (''cogito, ergo sum'')
    \item ein Körper ist ausgedehnt, füllt den Raum aus (res extensa), "ich bin ein denkendes Ding" (res cogitans)
    \item Kritik an Sinneswahrnehmung (Beispiel von Wachs, das zerfließt)
\end{itemize}
\subsection*{Mediation III}
\begin{itemize}
    \item will sich ganz auf sich selbst konzentrieren, schließt Außenwelt völlig aus
    \item erkennt an eigener Unvollkommenheit, dass es Volkommenes geben muss
\end{itemize}
$\rightarrow$ ontologischer Gottesbeweis (schon bei Aristoteles und Thomas von Aquin): Gott als höchste
Volkommenheit, die erste Ursache, den unbewegten Beweger

$\rightarrow$ für Descartes ist unzweifelhaft, dass es ihn selbst und Gott gibt
\section*{John Locke (Sensualismus)}
Nihil es in intellectam, quod non erat in sensam

Nichts ist im Geist, was vorher nicht in den SInne war

Mensch ist Zunächst ein blankes Blatt Papier
Zwei Schritte führen nach Locke zu Erkenntnis
\begin{itemize}
    \item Sensation: Der GEist nimmt auf, was die Sinne ihm liefern
    \item Reflexion: Eigentätigkeit des Geister: denken, zweifeln, schließen, erlernen
\end{itemize}
Wahrnehmungen werden verarbeitet und kombiniert.
In der Entwicklung kommt die Sensation vor der Reflexion. Wichtig ist auch das Molyneaux Problem. (Blinder und Kugel)

\section*{Kant}
Kant gebrauchte die traditionelle Unterscheidung zwischen analytischen und synthetischen Urteilen. Ein analytisches Urteil erläutert ein Wort.
(Bsp. Die Kugel ist rund) Synthetische Urteil gehen darüber hinaus. Er führte zudem die Begriffe von Erkenntnissen \emph{a priori} und 
\emph{a posteriori}. Man gelangt durch reines Nachdenken zu einer Erkenntnis a priori. Eine Erkenntnis a posteriori ist eine Erkenntnis
aus Erfahrung.

Nach Kant sind synthetische Urteile a priori möglich.

Erkenntnis ist nach Kant eine synthese aus Erfahrung und Begriffen. Ohne Sinne könnten wären wir uns keines Gegenstands bewusst.
Ohne Verstand könnten wir uns kein Bild von ihnen bilden.

Nach Kant ist jedem Raum und Zeit als Anschauungen a priori von Geburt an gegeben. GLeichzeitig wird unser Geist von Kategorien des Denkens bestimmt,
welche unsere Wirklickeitserfassung strukturieren. DIese Sind:
\begin{center}
    \begin{tabular} {c c c c}
        Quantität & Qualität & Realität & Modalität \\
        Einheit & Realität & Substanz & Möglichkeit \\
        Vielheit & Negation & Kausalität & Dasein \\
        Allheit & Limitation & Wechselwirkung & Notwendigkeit
    \end{tabular}
\end{center}
Aber auch Kant setzte der Erkenntnis Grenzen. Es ist dem Menschen nicht möglich über die Welt der Erscheinung (phainomina) hinauszugehen.
Die reale Welt (noumena), das Ding an sich sei nicht durch die Vernunft erkennbar. Metaphysische Aspekte wie Seele, Unsterblichkeit und
Gott sind alle nicht erkennbar.

Kant kritisiert am Empirismus, dass alle Erfahrung apriorische Formen vorraussetzen. Gegen den Rationalismus spricht,
dass apriorische Formen meist auf Anschauung bezogen sind.

Die transzendale Erkenntsnis ist nach Kant eine Erkentnis über die Grenzen der Erkenntis. Da das Ich empirisch ist, ist die 
Grenze unserer Wahrnehmung genau wie ein Auge nicht sich selbst wahrnehmen kann auf sich selbst beschränkt.
\end{document}