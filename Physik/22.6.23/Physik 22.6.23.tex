\documentclass{article}
\usepackage{amsmath}

\title{Physik - Induktion und Schwingungen}

\begin{document}
\maketitle

\section*{Formelsammlung}
\subsection*{Gemischt}
\begin{gather*}
    F = m * g\\
    \text{Frequenz in Hertz } f= \frac{1}{T}
\end{gather*}
\subsection*{Induktion}
Grundformel:
\begin{equation*}
    U_i = -n * \dot{\Phi}
\end{equation*}

\begin{enumerate}
    \item \begin{gather*}
              U_i = -n * B * \dot{A}\\
              U_i = -n * B * d * v
          \end{gather*}
    \item \begin{gather*}
              U_i = -n * A * \dot{B}
          \end{gather*}
          Berechnung des Magnetfelds einer Spule:
          \begin{gather*}
              B = \mu_0 * \mu_r * \frac{N}{l} * I\\
              \dot{B} = \mu_0 * \mu_r * \frac{N}{l} * \dot{I}
          \end{gather*}

    \item Lenzsches Gesetz:
          \begin{gather*}
              I = \frac{U}{R}\\
              F_L = n * B * I * s
          \end{gather*}
\end{enumerate}

\subsection*{Schwingungen}
Federpendel:
\begin{equation*}
    T = 2 * \pi * \sqrt{\frac{m}{D}}
\end{equation*}
Fadenpendel:
\begin{equation*}
    T = 2 * \pi * \sqrt{\frac{l}{g}}
\end{equation*}

Allgemeines:
\begin{gather*}
    \omega = \frac{2*\pi}{T}\\
    s(t) = \hat{s} * sin(\omega*t)\\
    \dot{s}(t) = \hat{s} * \omega * cos(\omega*t)\\
    \ddot{s}(t) = \hat{s} * \omega^2 * -sin(\omega*t)
\end{gather*}
Start im ausgelenktem Zustand:
\begin{gather*}
    s(t) = \hat{s} * cos(\omega*t)\\
    v(t) = \hat{s} * \omega * -sin(\omega*t)\\
    a(t) = \hat{s} * \omega^2 * -cos(\omega*t)
\end{gather*}

\begin{gather*}
    W_{Spann} = \frac{1}{2} * D * s^2\\
    W_{Kin} = \frac{1}{2}  m * v\\
    W_{pot} = m*g*h
\end{gather*}

\section*{Schwingungen}
Eine Bewegung eines Körpers ist dann periodisch, wenn er in regelmäßigen
Abständen in den gleichen Bewegungszustand Zurrückkehrt. Schwingungen sind
spezielle periodische Bewegungen. Damit eine periodische Bewegung eine Schwingung
ist muss gelten:
\begin{enumerate}
    \item Es existiert eine so genannte Ruhelage, in der die Kräfte auf
    dem Körper im Gleichgewicht sind. Die Ruhelage ist also die nicht ausgelenkte Position.
    \item  Ist ein Körper ausgelenkt wirkt auf ihn eine Rückstellkraft, die den Körper in Richtung Ruhelage beschleunigt.
\end{enumerate}
Eine Schwingung ist harmonisch, wenn sie sich mit einer Sinusfunktion beschreiben lässt.
\subsection*{Grundbegriffe}
\begin{itemize}
    \item Periodenzeit T: Dauer bis der gleiche Bewegungszustand erreicht wird
    \item Amplitude $\hat{s}$: Maximale Auslenkung von der Ruhelage
    \item Auslenkung/Elongation s: Entfernung von der Ruhelage
    \item Frequenz f: Anzahl an Perioden die Sekunde
\end{itemize}

\section*{Induktion}
Der Magnetische Fluss $\Phi = A * B$ ist ein Maß dafür, wie viele Feldlinien des Magnetischen Felds durch eine Leiterschleife laufen.
Ändert sich der Magnetische Fluss durch eine Leiterschleife so wird in der Schleife ein Induktionsstrom induziert.

\end{document}
