\documentclass{article}
\usepackage{amsmath}

\title{Physik - Induktion und Schwingungen}

\begin{document}
\maketitle

\section*{Formelsammlung}
\subsection*{Gemischt}
\begin{gather*}
    F = m * g\\
    \text{Frequenz in Hertz } f= \frac{1}{T}
\end{gather*}
\subsection*{Induktion}
Grundformel:
\begin{equation*}
    U_i = -n * \dot{\Phi}
\end{equation*}

\begin{enumerate}
    \item \begin{gather*}
              U_i = -n * B * \dot{A}\\
              U_i = -n * B * d * v
          \end{gather*}
    \item \begin{gather*}
              U_i = -n * A * \dot{B}
          \end{gather*}
          Berechnung des Magnetfelds einer Spule:
          \begin{gather*}
              B = \mu_0 * \mu_r * \frac{N}{l} * I\\
              \dot{B} = \mu_0 * \mu_r * \frac{N}{l} * \dot{I}
          \end{gather*}

    \item Lenzsches Gesetz:
          \begin{gather*}
              I = \frac{U}{R}\\
              F_L = n * B * I * s
          \end{gather*}
\end{enumerate}

\subsection*{Schwingungen}
Federpendel:
\begin{equation*}
    T = 2 * \pi * \sqrt{\frac{m}{D}}
\end{equation*}
Fadenpendel:
\begin{equation*}
    T = 2 * \Pi * \sqrt{\frac{l}{g}}
\end{equation*}

Allgemeines:
\begin{gather*}
    \omega = \frac{2*\pi}{T}\\
    s(t) = \hat{s} * sin(\omega*t)\\
    \dot{s}(t) = \hat{s} * \omega * cos(\omega*t)\\
    \ddot{s}(t) = \hat{s} * \omega^2 * -sin(\omega*t)
\end{gather*}
Start im ausgelenktem Zustand:
\begin{gather*}
    s(t) = \hat{s} * cos(\omega*t)\\
    v(t) = \hat{s} * \omega * -sin(\omega*t)\\
    a(t) = \hat{s} * \omega^2 * -cos(\omega*t)
\end{gather*}

\begin{gather*}
    W_{Spann} = \frac{1}{2} * D * s^2\\
    W_{Kin} = \frac{1}{2}  m * v\\
    W_{pot} = m*g*h
\end{gather*}

\end{document}
