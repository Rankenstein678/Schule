\documentclass{article}

\begin{document}
\section*{Koloialismus - Unterwerfung der Welt durch Europa}
\begin{itemize}
    \item Imperialismus:\\
        Bestreben von Staaten politisch, wirtschaftlich und militärisch Macht auf Gebiete außerhalb ihrer Landesgrenzen auszuüben.
    \item Kolonialismus:\\  Unterwerfung und Herrschaft über fremde Gebiete
    \item Dekolonisation:\\
        Die Auflösung der europäischen Kolonialreiche im Zeitraum 1945-1975
\end{itemize}
Aspekte des Kolonialismus:
\begin{itemize}
    \item Aufzwingung westlicher Werte (''zivilisatorische Überlegenheit'')
    \item Landraub
    \item Gewaltherrschaft/Vernichtung
    \item Rassismus / pseudowissenschaftliche Rsssenlehre
    \item Entwürdigung / Entmenschlichung
    \item Ausbeutung zwecks europäischen Konsums (Kolonialwaren wie Tee, Gewürze, Zucker, etc.)
\end{itemize}
\section*{Dekolonisation}
\subsection*{Phasen}
\subsubsection*{
    1. Befreiungsrevolutionen in den 1770er und 1820er Jahren in Nord und Südamerika
}
Diese Revolutionen wurden mit militärischen Freiheitskämpfen begangen und endeten in der Gründung
eigenständiger Republiken. Beispiele: USA 1776, Haiti 1790, Lateinamerika 1820er
\subsubsection*{
    2. Almähliche und überwiegend gewaltfrei verlaufende Ausweitung politischer Spielräume in den Siedlungskolonien
    des British Empire zu Beginn des 10. Jahrhunderts.
}
z.B. Kanada, Australien, Neuseeland und Südafrika als ''white dominions'' (sich selbst regierende Staaten innerhalb des Empire)
\subsubsection*{3. Die eigentliche Dekolonisation des 20. Jahrhunderts nach 1945}
Teilweise gewaltfrei, teilweise gewaltsame ''Revolution von unten'' 

Beispiele sind Indien und Pakistan, Algerien, Marokko, Südsahara Staaten im ''Afrikanischen Jahr'' 1960 (17 Staaten werden unabhängig)

\subsection*{Ursachen}
\begin{itemize}
    \item wachsende Ausbeutung der Kolonien (Rohstoffe, aber auch Soldaten) während des ersten und zweiten
    Weltkriegs insbesondere durch Frankreich und Großbritannien
    \item Bewusstseinswandel durch das in den ''Vierzehn Punkten'' (1918) des US-Präsidenten Woodrow Wilson verkündete
    ''Selbstbestimmungsrecht der Völker''
    \item Gründung der UNO (1945) und die Verpflichtung ihrer Mitglieder auf das PRinzip der souveränen
    Gleichheit aller Mitgliedsstaaten
    \item Bewusstseinswandel für imperiale Unterdrückung durch die kommunistische Russische Revolution und den Sturz
    des Zarenreichs (1917)
    \item Oropagierung des Befreiungskampfs unterdrückter Völker durch kommunistische Parteien und liberal gesinnte Bürgerrechtsvewegungen
\end{itemize}
\subsection*{Europäische Rechtfertigung}
Meist Zivilisationserzählung: Europäer als Zivilisationsbringer und ''Lehrer der Wilden''

\subsection*{Folgen}
Die Meinungnen scheiden sich über die Folgen der Kolonisierung.
\subsubsection*{Europäische Verantwortung}
Viele Probleme der Gegenwart sind direkte Folgen des Kolonialismus. Beispielsweise liegen die Ursprünge des Dschihad in ehemlaigen Kolonien,
die Mehrzahl der Migranten stammt aus den ehemaligen Kolonien, die Ursache der Probleme der ehemaligen Kolonien sind va. die Hinterlassenaschaften
der Kolonialmächte.
Def. Kolonialismus:

Kolonialismus ist allgemein ein Ausplünderungs/Unterdrückungsverhältnis

\subsubsection*{Eigenverantwortung}
Die Probleme der Gegenwart lassen sich nicht als direkte Folge des Kolonialismus verstehen.
Der Dschihad ist eine Reaktion auf die Moderne. Flucht und Migrationsströme lassen sich auf vielfältige regionale und historische
Unterschiede zurückführen. Außerdem nahmen viele Kolonien komplett verschiedene Entwicklungswege (Simbabwe vs Botswana)
Def. Kolonialismus

Kolonialismus ist eine Bezeichnung für eine historische Epoche.
\end{document}