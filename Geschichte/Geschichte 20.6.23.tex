\documentclass{article}
\title{Geschichte}
\begin{document}
\maketitle
\section*{Notes}
\begin{itemize}
    \item Es heißt BURGEOISIE
\end{itemize}
\section*{Dreieck der Alternativen}
Drei Alternativen standen Gesellschaften im 19. Jahrhunder offen:
\subsection*{NS ("Rassenmodell")}
Der NS zeichnete sich durch folgende Punkte aus:
\begin{itemize}
    \item Parteiverbote
    \item Machtdemonstrationen
    \item Rassismus / "Rassenhass"
    \item Antisemitismus
    \item gewaltsame Unterdrückung andersdenkender
    \item Expansionismus
    \item Einen neuen Menschen hervorbringen
\end{itemize}
\subsection*{Stalinismus ("Klassenlose Gesellschaft)}
\begin{itemize}
    \item Einen neuen Menschen hervorzubringen als Ziel
    \item Totalitarismus
    \item Planwirtschaft
    \item gewaltsame Unterdrückung andersdenkender
    \item marxistische Ideologien
    \item Personenkult
\end{itemize}
\subsection*{Liberales Modell des Westens ("diverses Modell")}
\begin{itemize}
    \item Individualismus
    \item Menschen und Bürgerrechte (Grundrechte)
    \item Marktwirtschaft
    \item Pluralismus
    \item Gewaltenteilung
    \item Öffentlichkeit / Medien
    \item Minderheitenschutz
    \item Toleranz
\end{itemize}
\section*{Europa nach dem Ersten Weltkrieg - Durchbruch der Demokratien und "Selbstbestimmungsrecht der Völker" / "14-Punkte-Wilsons"}

\begin{itemize}
    \item 1914-1918 - Der erste Weltkrieg stellte eine noch nie
          dagewesene Katastrophe dar, zerstörte weite Teile
          Europas und traumatisierte viele Menschen -> ca. 15-17 Mio. Totoe
    \item 1917 - Das russische Zarenreich wird kommunistisch
    \item 1918 - Die Siegermächte, vor alle die USA forden:
          "Make the world safe for democracy" = Einführung von
          Demokratien als Garant des Friedens

          Selbstbestimmungsrecht der Völker als Norm
          ("14-Punkte" Wilsons (Grundzüge einer Friedensordnung,
          die der Us. Präsident Wilson in einer Rende vor dem
          Kongress hielt))

    \item 1918 - 1919 - Zahlreiche neue Nationalstaten entstehen, z.B.
          Polen, Ugarn, Tschechoslowakei u.a

\end{itemize}
Jedoch sind bis heute Begriffe wie "Volk" und "Nation" unklar.
Außerdem bietet das Selbstbestimmungsrecht der Völker ein Alibi
für scheinbare Homogenität, die für Minderheiten Unterdrückung
bedeutet. Denn der Kern der Zivilisation ist Heterogenität mit
Bürgerrechten, nicht Homogenität

\section*{Leitbild "Sowjetkommunismus" - der "neue Mensch" und die klassenlose Gesellschaft}
Die ideologischen Grundlagen des Sowjetkommunismus waren die
Überlegungen Karl Marxs. Nach Marx befand sich die Welt in
einem konstanten Klassenkampf zwischen den Besitzern der
Produktionsmittel, auch Bourgeoisie genannt" (in unserer Zeit Unternehmern) und den
dem Proletariat (heutzutage Arbeiter). Die Schere zwischen
beiden wird sich immer weiter öffnen, bis die Lebensumstände
so schlecht werden, dass es zu einer Revolution des Proletariats
kommt. Das Proletariat wir dann zur neuen Bourgeoisie. Um diesen
Kreislauf zu durchbrechen schlug er den Kommunismus vor.
\\Auf dieser Ideologie basierte der Sowjetkommunismus. Zuerst
sollte der Kapitalismus durch ein sozialistische Revolution
überwunden werden $\rightarrow$ Lenin führte mit einer Kaderpartei
eine Revolution herbei.  Danach sollte nach der Einführung der
"Diktatur des Proletariats" als Zwischenstufe mit dem Kommunismus
der Klassenkampf überwunden werden.

Der "neue Mensch" des Sowjetkommunismus war das angestrebte Ideal
\begin{itemize}
    \item Wissenschaft statt Religion
    \item Kontrolle statt Gefühlen, Instinkten
    \item kollektiv (Gemeinsinn) statt Egoismus (Eigensinn)
    \item Homo Sapientissimus als Grundlage einer perfekten Gesellschaft
\end{itemize}
$\rightarrow$ Antiindividualistisches Menschenbild

\section*{ Demokratie auch in Deutschland? Die Novemberrevolution}
\begin{itemize}
    \item Oberste Heerleitung fordert Waffenstillstandsverhandlungen (auch um Schuld auf Regierung abzuwälzen)
    \item Aufgrund der Forderungen Wilsons überträgt der Kaiser
          seine Macht an den Reichskanzler Prinz Max von Baden $\rightarrow$ parlamentarische Monarchie
    \item Es kommt zum Kieler Matrosenaufstand; die Aufstände breiten
          sich in ganz Deutschland aus $\rightarrow$ Novermberrevolution
    \item Reichskanzler ruft Rücktritt des Kaisers aus;
          Phillip Scheidemann (MSPD) ruft die "deutsche Republik" aus;
          Karl Liebknecht (USPD) die "sozialistische Republik" $rightarrow$ Systemwechsel
    \item "Rat der Volksbeauftragten" aus 3 MSPD und 3 USPD
          Mitgliedern bildet eine provisorische Regierung
    \item Bündniss Ebert-Groener (Groener: General des Militärs) beschließt
          gemeinsames vorgehen gegen radikale kommunistische Kräfte
    \item Reichsrätekongress spricht sich für parlamentarische Republik aus
    \item Kommunisten um Rosa Luxemburg und Karl Liebknecht versuchen Umsturz (Spartakusaufstand)
          Der Aufstand wird unter SPD Führung brutal von Freikorps niedergeschlagen
          Rosa Luxemburg und Karl Liebknecht werden ermordet.
    \item Wahl zur Nationalversammlung, die eine Verfassung erarbeitet
\end{itemize}
$\rightarrow$ Revolution von Oben, Gewaltfrei, Alte Eliten beeinflussen die Regierung
\subsection*{Akteuere}
\begin{itemize}
    \item MSPD \begin{itemize}
              \item parlamentarische Regierungungsform
              \item Angst vor Gewalt, Chaos und Bürgerkrieg
              \item Unterstützt von Mehrheit der Arbeiter
              \item Kontakte zu anderen bürgerlichen Parteien
          \end{itemize}
    \item USPD \begin{itemize}
              \item gegen parlamentarische Regierung
              \item Macht bei Arbeiter - und Soldatenräten
              \item Minderheit
          \end{itemize}
    \item Andere Parteien \begin{itemize}
              \item Zentrum + Liberale halten sich zurück aber stehen parlamentarischer Regierung positiv gegenüber
              \item Revolutionsgegner fürchten radikalen Umsturz, zunächst passiv
              \item radikale Revolutionsgegner bilden Freikorps aus ehemaligen SOldaten
          \end{itemize}
\end{itemize}
\section*{Die erste deutsche Republik}
\subsection*{Belastung durch den Versailler Vertrag}
Der Versailler Vertrag sprach Deutschland die alleinige Kriegsschuld zu. Außerdem kam es
für Deutschland zu einem GEbietsverlust von 13\%. Dazu kamen noch unbegrenzte
Reparationszahlungen und eine Redukzion der Reichswehr auf 100.000 Mann und keine Flotte.
\\Dadurch wurde der Versailler Vertrag auch als "Diktatfrieden" bezeichnet.
Er enthielt ausßerdem keine Bestimmungen zum wirtschaftlichen Wiederaufbau
und ordnete eine terretoriale Neuordnung an, die die Zerissenheit in Europa noch vergrößerte.
Dadurch bot die Revision des Vertrags Populisten ein gutes Propagandaziel.
\subsection*{Die Weimarer Reichsverfassung}
\begin{itemize}
    \item Allgemeines, gleiches, unmittelbares, geheimes Verhältnisswahlrecht ohne Sperklausel \begin{itemize}
              \item Erstmals waren Frauen wahlberechtigt
              \item Viele kleine Parteien was zur Zersplitterung und keinen langristigen Koalitionen führte
          \end{itemize}
    \item Reichspräsident kann Reichstag auflösen $\rightarrow$ parlamentarische Kontrolle kann durch RP aufgehoben werden
    \item RP wird direkt vom Volk gewählt $\rightarrow$ Besondere Legitimation als plebiszitäres Element
    \item RP kann bei Gefährdung mit Zustimmung des RT Notverodnungen erlassen $\rightarrow$ quasi diktatorische Vollmacht des RP
    \item RP ernennt und entlässt RK ohne Mitwirkung des RT $\rightarrow$ RK abhängig von Wohlwollen des RP
    \item Volksentscheid bei Antrag durch RP $\rightarrow$ pol. Demagogie möglich.
    \item Gleichberechtigung von Frauen und Männern
\end{itemize}
\subsection*{Das Krisenjahr 1932}
\begin{itemize}
    \item Ruhrbesetzung der Franzosen löst Hyperinflation aus
    \item Der Mittelstand verliert Ersparnisse und radikalisiert sich
    \item Putschversuche von Links und Rechts
    \item 9.November Hitlerputsch in München
\end{itemize}
\subsection*{Die Goldenen 20er}
\begin{minipage}[t]{.5\textwidth}
    \begin{itemize}
        \item [\textbf{Blütejahre:}]
        \item Freizeitaktivität
        \item Kultur, Theater
        \item Kino, Funk, Massenmedien
        \item Mutterschutz, Kinderhort
        \item Mobilität, Automobil
        \item Freizügigkeit
        \item Sozialpolitik
        \item Mode
    \end{itemize}
\end{minipage}%
\begin{minipage}[t]{.5\textwidth}
    \begin{itemize}
        \item [\textbf{Scheinblüte:}]
        \item scheinbare Gleichberechtigung
        \item Arbeitslosigkeit
        \item Wohnungsnot, Elendsquartiere
        \item Fließbandarbeit
        \item (sittenlosigkeit)
        \item Propaganda, Hetze möglich durch Massenmedien
    \end{itemize}
\end{minipage}%
\section*{Das Ende der Weimarer Republik}
\begin{itemize}
    \item Zunehmende Radikalisierung (bei Wahlen) nach links und vorallem Rechts
    \item Erstarken der alten Eliten (Lehrer, Richter, Generäle, Unternehmer, Adlige), die die NSDAP unterschätzten
    \item Weltwirtschaftskrise mit 6 Mio. Arbeitslosen, Hungerkrise
    \item Demokratiefeindliche Rolle des 2. RP Hindenburgs, der Hitler nicht verhinderte
    \item Uneinigkeit der demokratischen Parteien (u.a Spaltung der Arbeiterschaft in KPD und SPD, fehlende Kompromissbereitschaft)
    \item mentale Belastung durch den Versailler Vertrag
    \item Schwächen der Verfassung die Präsidialkabinette ermöglichten
\end{itemize}
$\rightarrow$ Demokratie ohne Demokraten


\section*{Die Errichtung der NS-Diktatur }

\subsection*{Ideologische Grundsätze}

\begin{minipage}[t]{.33\textwidth}
    \begin{itemize}
        \item [\textbf{Rasse:}]
        \item Menschen gehören verschieden wertigen Gruppen an
        \item Eine Kreuzung der Gruppen führt zum körperlichen und geistigen Niedergang der höheren Gruppe
    \end{itemize}
\end{minipage}%
\begin{minipage}[t]{.33\textwidth}
    \begin{itemize}
        \item [\textbf{Arier:}]
        \item höchstmöglichste "Rasse", die durch Vermishcung ihren Platz im Paradies verliert
        \item haben volle Opferbereitschaft (kein Egoismus) $\rightarrow$ Vergleich sowjetkommunistischer neuer Mensch
        \item kulturschaffend kreativ
    \end{itemize}
\end{minipage}%
\begin{minipage}[t]{.33\textwidth}
    \begin{itemize}
        \item [\textbf{Jude:}]
        \item niedrigste "Rasse"
        \item stärkster Selbsterhaltungstrieb
        \item fehlernder Idealismus
    \end{itemize}
\end{minipage}

\subsection*{Verfolgung und Terror}
\subsubsection*{Gleichschaltung}
Alle Lebensbereiche sollten unter die Macht der Nationalsozialisten gestellt werden.
Wer sich wiedersetzte wurde bedroht und/oder verfolgt.

\subsubsection*{Schutzhaft}
\begin{itemize}
    \item Vor 1933: Schutzgewahrsam, Vorbeugehaft
    \item Nach 1933/Reichsbrandverodnung: Die Geheime Staatspolizei (Gestapo) hat nun
    Raum für jegliche Formen von Willkür gegen politische und ideologische Gegner.
\end{itemize}
\subsubsection*{Zielgruppen des NS-Terrors}
1.Phasse 1933/34
\begin{itemize}
    \item gegen politische und religiöse Gegner
\end{itemize}
2.Phase ab 1935/36
\begin{itemize}
    \item Bündelung des Terrors durch Himmler und Heydrich
    \item richtete sich nun gegen Menschen, die Verbrechen gegen die Volksgemeinschaft begingen
    \item und Fremde
    \item und ideologische Feinsbilder
\end{itemize}
\end{document}